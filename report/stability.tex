\section{Stability analysis}

In this section we perform a linear stability analysis of macroscopic model with logistic term.
We will linearize around the homogeneous states, i.e. we consider the constant steady states $\bfA$ and $\bfB$ which satisfy:
	\begin{equation}
\begin{cases}
\p_t f^{A}-  \nabla \cdot (f^A\nabla_x(\Phi^{AA}* f^A)) - \nabla \cdot (f^A \nabla_x \Phi^{AB}*f^B) - D_A \Delta_x f^A - \nu^{A}f^A\left( 1-\frac{f^A+f^B}{f^{*}} \right)=0 \\

\p_t f^{B}-  \nabla \cdot (f^B\nabla_x(\Phi^{BB}* f^B)) - \nabla \cdot (f^B \nabla_x \Phi^{BA}*f^A) - D_A \Delta_x f^A - \nu^{A}f^B\left( 1-\frac{f^A+f^B}{f^{*}} \right)=0
\end{cases}
\end{equation}

Since $\bfA,\bfB$ do not depend on time and space, all derivative terms are zero and it leads us to following equations:
$$
\begin{cases}
	\nu^{A}f^A\left( 1-\frac{f^A+f^B}{f^{*}} \right)=0 \\
	\nu^{A}f^A\left( 1-\frac{f^A+f^B}{f^{*}} \right)=0
\end{cases}
$$
The first equation is satisfied either when $\bfA=0$ or $f^A\left( 1-\frac{f^A+f^B}{f^{*}} \right)=0$ and the second one is satisfied by $\bfB=0$ or $f^B\left( 1-\frac{f^A+f^B}{f^{*}} \right)=0$. In this case we obtain the following relation:  
$$ \bfA+\bfB=f^* $$
which means that, at the homogeneous states, we have reached the maximum carrying capacity. 
We perform a stability analysis, using perturbation term and Fourier transform, in order to understand if a possible perturbation can have effects on our model and it corresponds to the appearance of some clusters. \\
We obtain the following system:
\begin{align}
\p_t \begin{pmatrix} \hat{f}^A \\ \hat{f}^B
\end{pmatrix}(y,t)=
\end{align}
\begin{align}
\begin{pmatrix} -|y|^2(2\pi\bfA\hPAA(y)+D_A)-\nu_{b}^A\frac{\bfA}{f^*} & -|y|^22\pi\bfA\hPAB(y)-\nu_{b}^A\frac{\bfA}{f^*} \\ 
-|y|^2\bfB\hPBA(y)-\nu_{b}^B\frac{\bfB}{f^*} & -|y|^2(2\pi\bfB\hPBB(y)+D_B)-\nu_{b}^B\frac{\bfB}{f^*} 
\end{pmatrix}
\begin{pmatrix} \hat{f}^A \\ \hat{f}^B
\end{pmatrix}(y,t)
\end{align}
\begin{align}
=M(y)\begin{pmatrix} \hat{f}^A \\ \hat{f}^B
\end{pmatrix}(y,t)
\end{align}
This is a linear ODE system, hence its solution is:
$$
\begin{pmatrix} \hat{f}^A \\ \hat{f}^B
\end{pmatrix}(y,t)=c_1(y)exp^{\la_1(y)t}u_1(y)+c_2(y)exp^{\la_2(y)t}u_2(y).
$$
We want to compare the critical value of model with logistic term and without it. We are interested in instability  that implies the appearance of cell aggregates.
\begin{equation}
s^{*}_{\nu_b,\nu_d \ne 0}=\frac{(24 D_A+c'^{AA})\nu_{b}^{B}\bar{f}^{B}+(24 D_B+c'^{BB})\nu_{b}^{A}\bar{f}^A}{\nu_{b}^{B}\bar{f}^B\tilde{c}'^{AB}+\nu_{b}^{A}\bar{f}^A\tilde{c}'^{BA}}
\end{equation}

with $\bar{f}^A$ and $\bar{f}^B$ constant steady states, not necessarily equal and 
$c'^{ST}=\frac{2\pi k^{ST} \bfA \nu_c^{ST} R^{4}}{\nu_d^{ST}}$, $S,T \in \{ A,B \}$.

We define $c'^{AA}=k_1 \bfA, c'^{BB}=k_2\bfB, c'^{AB}=k_3\bfA, c'^{BA}=k_4\bfB$.
By $\bar{f}^B=f^*-\bar{f}^A$ we rewrite the following $s^*$:

\begin{equation}
s^{*}_{\nu_b,\nu_d \ne 0}=\frac{(24 D_A+k_1\bfA)\nu_{b}^{B}(f^*-\bfA)+(24 D_B+k_2(f^*-\bfA))\nu_{b}^{A}\bar{f}^A}{\nu_{b}^{B}(f^*-\bfA)k_3\bfA+\nu_{b}^{A}\bar{f}^Ak_4(f^*-\bfA)}
\end{equation}

To simplify notation, we take into account the following function depending on $\bfA$ and its derivative :
$$F(\bfA)=\frac{\a \bfA + \b (\bfA)^2 + \g}{\d \bfA + \eps(\bfA)^2}, \quad\quad 
\frac{\p F(\bfA)}{\p \bfA}=\frac{(\bfA)^2(\b \d-\a \eps)-2\eps \g \bfA - \g\d}{(\d \bfA+\eps (\bfA)^2)^2}$$ 
with parameters 
\begin{align}
\a= 24D_B \nu_b^A -24 D_A \nu_b^B +k_1 \nu_b^B f^* + k_2 \nu_b^A f^*, \quad
\b=-k_1 \nu_b^B -k_2 \nu_b^A, \\
\g= 24 D_A \nu_b^B f^*, \quad 
\d= k_3\nu_b^B f^*+k_4 \nu_b^A f^*, \quad
\eps= -\nu_b^B k_3-	\nu_b^A k_4.
\end{align}

\begin{remark}
	The condition $\d \bfA + \eps (\bfA)^2 \ne 0 $ ensures the existence of function, in other words we get  $\bfA \ne 0 $ or $\bfA \ne -\frac{\d}{\eps}=f^*$. 
\end{remark}

We are looking for the zero points of $\frac{\p F(\bfA)}{\p \bfA}$, i.e. 
$ \bfA=\frac{2 \eps \g \pm \sqrt{\Delta}}{2(\b \d - \a \eps )} $ with the discriminant
$\Delta=4\eps^2 \g^2 + 4 \g \b \d^2 -4 \g \d \a \eps = 4\eps^2(24D_A \nu_b^B f^*)(24 D_B \nu_b^A f^*)$. After computations we can conclude:
\begin{equation}
\bfA=\frac{f^*(D_A \nu_b^B \pm \sqrt{D_A D_B \nu_b^B \nu_b^A})}{D_A \nu_b^B -\nu_b^A D_B}.
\end{equation}

If we look at those two values, we can find that:

\begin{equation}
 0<\frac{f^*(D_A \nu_b^B - \sqrt{D_A D_B \nu_b^B \nu_b^A})}{D_A \nu_b^B -\nu_b^A D_B}<f^*.
\end{equation}
and
\begin{equation}
 \left|\frac{f^*(D_A \nu_b^B + \sqrt{D_A D_B \nu_b^B \nu_b^A})}{D_A \nu_b^B -\nu_b^A D_B}\right|>f^*.
\end{equation}






