\section{Mathematical model}
The starting point for this work is the model previously proposed in \textbf{Citation}. The authors describe two families of particles referred to as type A and tyoe B and it is assumed that the cell population not depends on time, then  $N_A, N_B$ number of particles are fixed. In this section we want to introduce and to investigate a simple modification of the model, i.e. the effect of cell division on the model. We will introduce and analyze the model with logistic growth term that take into account birth and death cells.

	\subsection{Microscopic model}
	For the microscopic model, we start from the model presented in \textbf{citation}, using the same dynamics. The equation of motion for each individual particle in the so-called overdamped regime, between two linking/unlinking events is:
	\begin{equation}\label{micro}
	\begin{cases}
	d X_i^{A}=-\mu \nabla_{X_{i}^{A}}W^{A}(X^{A},X^{B})dt + \sqrt{2D_{A}} d B_{i}, \quad \forall i \in\{1, \dots, N_{A}\}
	\\
	d X_i^{B}=-\mu \nabla_{X_{\l}^{A}}W^{B}(X^{A},X^{B})dt + \sqrt{2D_{B}} d B_{\l}, \quad \forall \l \in \{1, \dots, N_{B}\}
	\end{cases}
	\end{equation}
	with $W^A(X^A,X^B)$ and $W^B(X^A,X^B)$ total energy of A-particles and B-particles defined as the sum over all pairwise link potentials acting respectively on particles A and on particles B. \\
	The main change in the model is to introduce a cell birth and death process. Our modeling is based on the birth and death process proposed in \textbf{citation}. The idea is that a cell of population of type $S$ has a probability $\beta_S$ to divide into two cells and a probability $\delta_S$ to die at each time step. This probability depends on the population size. Here we add also a spatial dependence to the probability rate:

	
%	-------
%The main change on the model is to introduce a cell birth and death process. We base the modelisation of the birth and death process on the one proposed in \textbf{citation}. The idea is that at each time step, a cell of population $S$ has a probability $\beta_S$ to divide into two cells and a probability $\delta_S$ to die. The probability depend on the size of the population. Here we add also a spatial dependencie for the probability rate:
%   ----------
	\begin{equation}
\beta_{S}(X_i^S)=b_{0}^{S}-(b_{0}^{S}-\theta_{S})\left(\frac{\mathcal{N}_{R_0}(X_i^S)}{N^{*}}\right), \quad\quad \delta_{S}(X_i^S)=d_{0}^{S}+(\theta_{S}-d_{0}^{S})\left(\frac{\mathcal{N}_{R_0}(X_i^S)}{N^{*}}\right)
\end{equation}
	where the coefficient $\mathcal{N}_{R_0}(X_i^S)$ is the number of cell (of both population) at distance $R_0$ of the cell located in $X_i^S$ and $N^*$ is the maximal number of cell in a radius $R_0$ allowing cell division. The coefficient $\theta$ must be taken in the range $d_{0}^{S}<\theta<b_{0}^{S}$.


	\subsection{Macroscopic model}
	The Macroscopic model presented in \textbf{citation} can be derived from the microscopic model in a large population assumption. Here, we modify this model, simply by adding a logistic term to the equations. We assume for the moment that the obtained model can also be derived from the microscopic one.
	
	\begin{equation}
		\begin{cases}
	\p_t f^{A}=  \nabla \cdot (f^A\nabla_x(\Phi^{AA}* f^A) + f^A \nabla_x( \Phi^{AB}*f^B)) + D_A \Delta_x f^A + \nu^{A}f^A\left( 1-\frac{f^A+f^B}{f^{*}} \right) \\
	
	\p_t f^{B}=  \nabla \cdot (f^B\nabla_x(\Phi^{BB}* f^B) + f^B \nabla_x (\Phi^{BA}*f^A)) + D_A \Delta_x f^A + \nu^{B}f^B\left( 1-\frac{f^A+f^B}{f^{*}} \right)
		\end{cases}
	\end{equation}
	where the function $\Phi$ correspond to an Hookean interaction potential:
	\begin{equation}
	\Phi^{ST}(x)=\frac{\nucST}{\nudST}\frac{\KST}{2}
	 \begin{cases}
	  (|x|-R)^2, \quad \text{for } |x|\leq R\\
	  0, \quad \text{for } |x|> R
	 \end{cases}
	\end{equation}
	with  $\ka^{ST}$ intra- and inter-species repulsion intesities already included in function.
	The logistic growth involve both cells of population $A$ and $B$ in the same way. The coefficient $f^*$ called the carrying capacity represents the maximum popolation size that can be present in the environment.
