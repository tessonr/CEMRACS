\section*{Introduction}
The starting point of this work was the model previously proposed in \textbf{Citation}. They provided a detailed multiscale analysis -- from a microscopic model to a macroscopic description, and its qualitative analysis -- of a system of particles interacting through a dynamical network. Indeed, their model describes point particles with local cross-links modeled by springs that are randomly created and destructed. The deduced in the mean field limit, assuming large number of particles and links that in the regime where the network evolution triggered by the linking/unlinking processes happens on a very short timescale, the link density distribution becomes a local function of the particle distribution density. The latter evolving on the slow time scale through an aggregation-diffusion equation, known also as the McKean-Vlasov equation. Their results have been extended and applied to the case of cell aggregation and segregation in [citation]. The aim of their work was to describe and explain the origin of cell aggregation and segregation during tissues morphogenesis. The ability of different cell types to segregate and aggregate is known to be a key process in many biological phenomenons as tissue differentiation especially in embryogenesis or tumor cells metastasis. 
	However, In their model, it was assumed that the cell population remain constant over the time, which means that there is no growth process. In this study, we investigate the effect of cell division on the aggregation and segregation process. Therefore, we derive (as rigorously as possible) in a first time a macroscopic logistic equation from the microscopic models of two species of cell populations introduced in [citation] by adding a local density-saturated growth process at the microscopic scale. The main difficulty of this first step is the varying number of the cells population due to the growth process. After the derivation of the macroscopic model, we perform stability analysis on the model and numerical simulations of the microscopic and the macroscopic model.